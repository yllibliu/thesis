\begin{titlepage}

\begin{center}
\large{摘要}
\end{center}

近年來,由於AI、VR產業的崛起,FPGA產業越來越受到重視。為了簡化FPGA的開發流程,使用嵌入式Linux會是一個不錯的方法。透過Linux Kernel提供的UIO驅動程式,我們可以把我們在硬體端設計出來的IP視為一個外部裝置,然後在Linux使用者空間裡的程式中,輕鬆地開發軟體端的應用。然而,有些硬體端的設計,卻無法透過同樣的方法,利用UIO驅動程式,建立裝置節點,而帶有直接記憶體存取IP的設計就是其中之一。由於UIO驅動程式並無法支援此種設計,我們必須擁有"root"	權限,才能使用我們的設計,但是提供"root"給一般使用者並不是一個好方法。在此論文中,我們修改了Linux內建的UIO驅動程式,使得一般用戶也能在使用者空間中使用帶有DMA的硬體設計。
%臣亮言:先帝創業未半,而中道崩殂;今天下三分,益州疲敝,此誠危急存亡之秋也。然侍衛之臣,不懈於內;忠志之士,忘身於外者:蓋追先帝之殊遇,欲報之於陛下也。誠宜開張聖聽,以光先帝遺德,恢弘志士之氣;不宜妄自菲薄,引喻失義,以塞忠諫之路也。宮中府中,俱為一體;陟罰臧否,不宜異同:若有作奸犯科,及為忠善者,宜付有司,論其刑賞,以昭陛下平明之治;不宜偏私,使內外異法也。侍中、侍郎郭攸之、費依、董允等,此皆良實,志慮忠純,是以先帝簡拔以遺陛下:愚以為宮中之事,事無大小,悉以咨之,然後施行,必得裨補闕漏,有所廣益。將軍向寵,性行淑均,曉暢軍事,試用之於昔日,先帝稱之曰“能”,是以眾議舉寵為督:愚以為營中之事,事無大小,悉以咨之,必能使行陣和穆,優劣得所也。親賢臣,遠小人,此先漢所以興隆也;親小人,遠賢臣,此後漢所以傾頹也。先帝在時,每與臣論此事,未嘗不嘆息痛恨於桓、靈也!侍中、尚書、長史、參軍,此悉貞亮死節之臣也,願陛下親之、信之,則漢室之隆,可計日而待也。


\vspace*{2em}
{關鍵字:} \emph{賽靈思,直接記憶體存取,AXI,Linux UIO驅動程式}

\end{titlepage}
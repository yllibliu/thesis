%\begin{page}

\begin{center}
\large{摘要}
\end{center}

近年來,由於AI、VR產業的崛起,FPGA產業越來越受到重視。為了簡化FPGA的開發流程,使用嵌入式Linux會是一個不錯的方法。透過Linux Kernel提供的UIO驅動程式,我們可以把我們在硬體端設計出來的IP視為一個外部裝置,然後在Linux使用者空間裡的程式中,輕鬆地開發軟體端的應用。然而,有些硬體端的設計,卻無法透過同樣的方法,利用UIO驅動程式,建立裝置節點,而帶有直接記憶體存取IP的設計就是其中之一。由於UIO驅動程式並無法支援此種設計,我們必須擁有"root"	權限,才能使用我們的設計,但是提供"root"給一般使用者並不是一個好方法。在此論文中,我們修改了Linux內建的UIO驅動程式,使得一般用戶也能在使用者空間中使用帶有DMA的硬體設計。

\vspace*{2em}
{關鍵字:} \emph{賽靈思,直接記憶體存取,AXI,Linux UIO驅動程式}

%\end{page}
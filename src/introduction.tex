\chapter{Introduction}
\label{cha:introduction}

%本章為簡易的LaTeX範本,使本文略微豐富些。由於編譯器是使用XeLaTeX,中英文可以無縫直接輸入,不需要額外繁瑣的環境設定。 Section~\ref{sec:simpletechnique1} 是撰寫文件時可以用的一些小技巧。 Section~\ref{sec:figandtable} 示範LaTeX繪製圖表的功能。

%-------------------------------------------------------------------------
% Section: 簡單範例和小技巧
%-------------------------------------------------------------------------

\section{Motivation}
\label{sec:Motivation}

%首先先示範一行簡單的數學式:
%\begin{equation} \label{eq:sample1}
%  y = \sum_{i=1}^{n}f(x_i),
%\end{equation}
%個人認為這比起用 Word 的方程式專業不少,在輸入上也方便得多。不過有些人可能會發現這條式子和一些常見的課本(像微積分)上的式子長得略微不同,這是因為他們額外安裝了付費的數學字型 MathTime Professional,約莫台幣三千元,可以讓 $y$ 和一些粗體希臘字母長得好看些,一些數學符號的間距也會被修正。

%不管是哪個科系,很多時候會需要把一些數據寫進論文裡。若是在論文撰寫的同時仍然有機會更動數據,那麼找出該數字出現的每個地方並一一改寫,這是一件很崩潰的事。所以小弟我的做法是把一些關鍵的數字寫在 \texttt{src/def.tex} 裡,利用 \texttt{newcommand} 去定義該數字。例如某項實驗做出來的準確度為 \expACC{}\%,那麼之後需要變動就只要改寫 \texttt{src/def.tex} 中的 \texttt{expACC} 即可。

%在編譯完一份文件,切莫忘記檢查 \texttt{.log} 檔案中是否含有 Overfull hbox 或 Underfull hbox 等訊息。前者代表存在一行文字無法左右貼齊,會有文字超出到頁邊空白處,後者則是在左右貼齊的過程中,文字間的留白過大。

%-------------------------------------------------------------------------
% Section: 圖表
%-------------------------------------------------------------------------

\section{Contribution}
\label{sec:contribution}

%表格當然可以用 Excel 匯出成 PDF 或圖檔在匯進來,但我是直接用 LaTeX 繪製,如 Table~\ref{tab:sample1}。圖也是一樣, Figure~\ref{fig:lena} 是一張影像處理常用的圖。但若是漂亮的數據圖繪製,那小弟我也是推薦使用 LaTeX 的向量繪圖功能。 Figure~\ref{fig:runtimeanalysis} 是用 PGFPlots 這個套件以存放在 \texttt{data/result.txt} 的實驗數據畫的圖,若是用 Matlab 畫得就比較快但比較醜。另外值得注意的是數據越多畫越久,上千個資料點可能會需要三秒到五秒不等的時間。

%\begin{table}
%\centering
%\begin{tabular}{clr|clr}
%  \toprule
%    \textbf{Id} & \textbf{Family} & \textbf{\#} & \textbf{Id} & \textbf{Family} & \textbf{\#} \\
%  \midrule
%  A & DroidKungFu   & 473 & K & FakePlayer      &  74 \\
%  B & DorDrae       & 420 & L & Wroba           &  74 \\
%  C & Meds          & 221 & M & Plankton        &  63 \\
%  D & Fakeguard     & 203 & N & DroidDreamLight &  52 \\
%  E & Boxer         & 202 & O & Cawitt          &  51 \\
%  F & Kmin          & 183 & P & Badao           &  46 \\
%  G & Rooter        & 117 & Q & Fake10086       &  46 \\
%  H & Boqx          & 114 & R & Cupi            &  39 \\
%  I & DroidAp       & 106 & S & Coogos          &  39 \\
%  J & DroidKungFu3  &  93 & T & DroidDream      &  39 \\
%  \bottomrule
%\end{tabular}


%\caption{Top 20 malware families in the dataset.}
%\label{tab:sample1}
%\end{table}

%\begin{figure}
%  \centering
%  \includegraphics[scale=0.5]{images/lena.png}
%  \caption[Short caption for Lena]{Long caption for Lena.}
%  \label{fig:lena}
%\end{figure}

%\begin{figure}
%\centering
%  \begin{tikzpicture}
%    \begin{axis} [
%      xlabel={Size of classes.dex (MiB)},
%      ylabel={Time (sec)},
%      xmin=0.01,
%      xmax=144,
%      xmode=log,
%      ymin=0.01,
%      ymax=144,
%      ymode=log,
%      height=10cm,
%      width=10cm,
%      grid=major,
%      legend cell align=left,
%      legend pos=south east,
%    ]
%    
%    \addplot [color=gray,only marks,mark size=1pt] table [x=size,y=time] {data/result.txt};
%    \addplot [color=black] coordinates {(0.01,0.0471) (10,55.5648)} node[below] at (0.7, 0.7) {$O(n^{1.024})$};

%    \legend{Samples, Estimation}

%    \end{axis}
%  \end{tikzpicture}
%  \caption{Runtime analysis of Androguard.}
%  \label{fig:runtimeanalysis}
%\end{figure}

%-------------------------------------------------------------------------
% Section: 參考文獻
%-------------------------------------------------------------------------

%\section{編輯參考文獻}
%\label{sec:ref}

%LaTeX 另個方便的地方就是參考文獻,我是使用 IEEEtran 的格式,它會依照出現的次序編排。若是使用 IEEEtranS 則會以作者姓氏為依據。另外需要注意它會雞婆地把書目標題的大小寫做改動,要保持原樣就得在 \texttt{reference.bib} 的標題加括弧,像是 \cite{permissionevolution} 和 \cite{permissionevolution2} 的差別。

%-------------------------------------------------------------------------
% Section: 總結
%-------------------------------------------------------------------------

%\section{總結}
%\label{sec:summary}

%用 LaTeX 寫論文或作業很方便,但速度快慢取決於對於 LaTeX 的熟悉程度,還不熟的同學可以搜尋大家來學LaTeX \cite{latex123},以入門來說是滿不錯的,祝各位碩博士生論文順利。




%-------------------------------------------------------------------------
% Section: 總結
%-------------------------------------------------------------------------
\chapter{Background technology}
\label{cha:Background technology}

%-------------------------------------------------------------------------
% Section: 總結
%-------------------------------------------------------------------------
\chapter{Proposed solution and evaluation}
\label{cha:proposed solution and evaluation}

%-------------------------------------------------------------------------
% Section: 總結
%-------------------------------------------------------------------------
\chapter{Discussion}
\label{cha:Discussion}

%-------------------------------------------------------------------------
% Section: 總結
%-------------------------------------------------------------------------
\chapter{Conclusion}
\label{cha:conclusion}



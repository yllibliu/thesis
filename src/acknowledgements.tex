\begin{titlepage}

\begin{center}
\large{誌謝}
\end{center}
研究所生活兩年,說長不長,一下就過去了,在這段期間中,最感謝的就是父母家人的支持,不管是經濟上的,還是心靈上的,他們都提供了最全面的援助,讓我能專心地完成學業。在研究的過程中,鄭振牟老師在我不順利的時候,總是能提出最核心、簡單的邏輯,幫助我理清思緒。bypeng學長在每meeting時,也都很耐心的聽我解釋遇到的問題,即使我自己都覺得進度緩慢,他還是不會特別催促我,相信我能自己掌握步調,讓我有更多空間能調配自己的進度。若沒有忠憲憲哥所做的github-repo,我的研究絕對沒辦法起步,而且對於軟體問題,憲哥總是能提供很有幫助的回饋,讓我對於軟體領域有更多理解。非常感謝JP學長在最後提供給我預口試的機會,也給我建議,幫我修改口試內容的架構,若沒有這個過程,我相信最後的口試表現,一定差非常多,真的很感謝!除了老師跟學長以外,唯一能了解我在做什麼,就只有同為FPGA組的遠哲了,在最後實驗的部分,若是沒有遠哲所提供的IP,我的實驗結果一定生不出來。昱嘉在我最後撰寫論文的時候,總是會來詢問我的進度,給我壓力(?),若沒有他,我的論文進度可能要再更往後拖。跟克烜雖然不同組,但是聊天起來還是很輕鬆,什麼都能聊,等八月玩完,九月再續攤吧!風聲大師紘賢沒什麼好說的,前來報到!要是沒有“掃LAB好夥伴”的鍾豪,我可能好幾次打掃都會開天窗,而且每次聽你報告的時候,都覺得聲音都太暖了吧,好羨慕。另外一位讓我好羨慕的是瑞智,每次臉書PO文都一堆讚,下面還一直推瑞智男神,太羨慕了!小白、怡慧,我都已經帶你們吃過那麼多店了,拜託不要再叫我想晚上要吃什麼了!然後我們說要打球,結果只打一次是不是太LOW了!呈翰、惇益、俊又,每次聽你們在說自己做的東西的時候,我都覺得你們好厲害,我想連我都能準時畢業了,你們一定也可以順利的完成自己的目標吧!加油!最後的最後,還是要再次感謝大家給我的幫助,不管大的小的,重要的不重要的,你們都構築了我兩年的精彩生活,再多的致謝都不足以表達我內心的感受,我想這時候還是套個國中國文課本的名句,要謝的人太多了,那就謝天吧!




\end{titlepage}
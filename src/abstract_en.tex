\begin{abstract}
In recent year, increasingly importance has been attached to FPGAs with the development of AI,VR. To simplify the development process on FPGAs, embedded Linux on FPGAs will be a good way. With UIO driver provided in Linux Kernel, we can mount our block design, that is, custom IP(Intellectual Property) core in Vivado as a device node, and program it in Linux user space. However, there are some designs that UIO driver cannot recognizes. The design with DMAs(Direct Memory Access) is the one of them. With this kind of design, because UIO driver is not work, we need "root" to control our IP, and providing root privileges to users is never a good solution. In this thesis, we modifiy UIO driver so that users can easily use designs with DMA in user-space.   
%An abstract is a brief summary of a research article, thesis, review, conference proceeding or any in-depth analysis of a particular subject or %discipline, and is often used to help the reader quickly ascertain the paper's purpose. When used, an abstract always appears at the beginning of a manuscript or typescript, acting as the point-of-entry for any given academic paper or patent application. Abstracting and indexing services for various academic disciplines are aimed at compiling a body of literature for that particular subject.

%The terms précis or synopsis are used in some publications to refer to the same thing that other publications might call an ``abstract''. In management reports, an executive summary usually contains more information (and often more sensitive information) than the abstract does.


\vspace*{2em}
{\bf Keywords:} \emph{Xilinx, DMA, AXI, Linux UIO driver}

\end{abstract}